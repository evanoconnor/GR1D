
\documentclass[10pt,nofootinbib]{article}
\usepackage{epsfig}
\usepackage{graphicx}
\usepackage{color}
\usepackage{amsmath}
\usepackage{amssymb}
\usepackage{url}
\usepackage[comma,numbers,sort&compress]{natbib}

\setlength{\textheight}{24cm}
\setlength{\oddsidemargin}{.5cm}
\setlength{\textwidth}{15.4cm}
\setlength{\topmargin}{-2cm}
\setlength{\parindent}{0cm}
\setlength{\parskip}{0.1cm}

\newcommand{\code}[1]{\texttt{#1}}

\title{\code{GR1D}:\\ An Open-Source Code for Stellar Collapse\\
 and Black Hole Formation}
\author{Evan\ O'Connor \& Christian\ D.\ Ott\\e-mail: {\tt GR1D@stellarcollapse.org}}
\begin{document}

%\maketitle
\begin{center}
\Large \bf GR1D v2.0 \rm \normalsize

 \bf Nov 20th, 2014 \rm

 \large An Open-Source Neutrino Radiation Hydrodynamics Code for
 Stellar Collapse and Black Hole Formation \normalsize

Evan\ O'Connor, North Carolina State University \& Christian\ D.\ Ott, California Institute of Technology

e-mail: {\tt GR1D@stellarcollapse.org}

\end{center}


\section{Disclaimer and Software License}
\code{GR1D} is open source and may be used at your own risk.  The code
comes with absolutely no warranty and we are unable to guarantee that
we will be able to provide support or help if you run into problems
running it. If you decide to use \code{GR1D} in published work, it is
YOUR responsibility to test the code and ensure its physical
correctness and consistency. 

{\bf If you find bugs or inconsistencies, please let us know via
  e-mail to \\ {\tt GR1D@stellarcollapse.org}\,!} This is particularly
probable if you are using \code{GR1D} in a way we have not thought of
yet.  We are also open for collaboration on extending \code{GR1D} to
include more physics. However, please avoid contacting us with
questions like: ``How do I get \code{GR1D} to read in my stellar
profile?'' Just remember that \code{GR1D} is plain Fortran 90 and that
basic knowledge of that language and the ability to use a text editor
will solve this problem for you.

While \code{GR1D} is open source, its copyright is held by Evan
O'Connor and Christian D.\ Ott. In the absence of suitable open
scientific software licenses, we release this version of \code{GR1D}
to the community under the {\bf Creative Commons}
\emph{attribution-noncommercial-share} alike license whose details can
be found at
\url{http://creativecommons.org/licenses/by-nc-sa/3.0/us}. Essentially,
you may use \code{GR1D}, but must make reference to our work, must not
use \code{GR1D} for commercial purposes, and any code including or
using our routines or part of them may be made publically available,
and if so, only under the same license.

When using \code{GR1D} in published work, please make reference
to the paper describing it:

\hspace{2cm}\parbox{12cm}{E.\ O'Connor and C.\ D.\ Ott, \emph{A New Spherically-Symmetric
General Relativistic Hydrodynamics Code for Stellar Collapse to Neutron
Stars and Black Holes}, Class. Quantum Grav., 2009 27 114103}

When using the neutrino transport model of \code{GR1D} in published
work, in addition to the above reference, please make reference to
the paper describing it:

\hspace{2cm}\parbox{12cm}{E.\ O'Connor, \emph{An Open-Source Neutrino
    Radiation Hydrodynamics Code for Core-Collapse Supernovae},
  \emph{eubmitted to The Astrophysical Journal: Supplement Series},
  2014 arXiv:1411.7058}

\section{Compiling:}
\subsection{make.inc}
All settings necessary for successful compilation of \code{GR1D} must
go into the file {\emph{make.inc}} which is read by the make
system. In particular, the compiler, compiler flages (e.g.,
optimization, debugging options, OpenMP usage for the transport), and
the location of external library include and library files must be
provided. \code{GR1D} is written in Fortran 90.  The compiler options
must be set in the {\it make.inc} file. The authors suggest using the
\code{MESA SDK} available at
\url{http://www.astro.wisc.edu/~townsend/static.php?ref=mesasdk}.

To activate support for finite-temperature EOS and neutrino leakage,
the following preprocessor definitions must be set in \emph{make.inc}:\\
{\tt HAVE\_NUC\_EOS=1}\\
{\tt HAVE\_LEAK\_ROS=1}\\
For the neutrino transport module you must have\\
{\tt HAVE\_LAPACK=1}\\
in addition to setting the LAPACK library above.

\subsection{External libraries}
\code{GR1D} requires a set of \code{HDF5} libraries and include files
compiled for use with your Fortran 90 compiler.  We use \code{HDF5}
libraries to write restart files and also to read in our EOS and
opacity tables.  The source files for \code{HDF5} are available from
the HDF5 Group at \url{http://www.hdfgroup.org/HDF5/} along with
straightforward installation instructions.  The path to the libraries
and include files must be set in the {\it make.inc} file.  This
library is included with the \code{MESA SDK} linked above.

\code{GR1D} also requires \code{LAPACK} for the matrix inversion and
the path must be set in the \emph{make.inc} file.  Here is the
\code{LAPACK} website, \url{http://www.netlib.org/lapack/}.  The
\code{LAPACK} libraries are also available in the \code{MESA SDK}

\subsection{Compiling \& Running}
Once all the settings have been set, running {\emph{$>$make}} will
compile the code.  If settings in {\emph{make.inc}} have changed,
first run {\emph{$>$make clean}} then {\emph{$>$make}}.  To run
\code{GR1D}, simply enter the command {\emph{$>$./GR1D}}. One can use
up to several cores to improve the wall time of \code{GR1D} when using
the transport.  set the environment variable \code{OMP\_NUM\_THREADS}
equal to 2,3,4 for varying levels of improvement.  As the calculation
can only be parallelized so much, there is no value in choosing a
large number of cores.
 
\section{{\it.short} initial data format}
\code{GR1D} uses the '.short' format for input stellar profiles. On
the first line of the input file is the number of lines in the input
file.  The rest of the file consists of lines with 8 columns each: (1)
zone index, (2) enclosed mass (g), (3) radial coordinate (cm), (4)
temperature (K), (5) density (g/cm$^3$), (6) radial velocity (cm/s),
(7) $Y_e$, and (8) $\Omega$ (rad/s). We include in this version of
\code{GR1D} an $n=3$ polytrope with a central density of
$5\times10^{10}$~g/cm$^3$ and total mass of 1.435~$M_\odot$. The
polytrope may be used for collapse calculations with the hybrid
EOS. For runs with finite-temperature EOS and neutrino leakage, a
presupernova stellar profile is necessary.  Various presupernova
models are available for download from stellar evolution groups,
(c.f. {\tt http://www.stellarcollapse.org/other\_resources} but note
different groups define variables differently, zone center or zone
interface values for example, {\emph{src/map\_profile.F90}} may need to
be adjusted accordingly). For example, there is a parameter setting
for the common \code{KEPLER} models that assumes the radial coordinate
is the outer edge of the zone rather than the center.  This can make a
large difference depending on the intial profile, be careful.

\section{Sample parameter files}
We include sample parameter files for test cases and some collapse
simulations.  These follow closely the simulations presented in our
\code{GR1D} methods paper, \cite{oconnor:10}, and include: (1)
Shocktube problems \#1 \& \#2, (2) Newtonian Sedov blast wave, (3)
Oppenheimer-Snyder Collapse of a pressureless dust ball, (4)
Hybrid-EOS core collapse of a progentior model, (5) Microphysical EOS
core collapse with neutrino leakage/heating of a progenitor
model. There are also sample parameter files for the neutrino
radiation transport that reproduce the results in \cite{oconnor:14}.

The sample parameter files live in the subdirectory 
\emph{sample\_parameter\_files} and must be copied into the
file \emph{parameters} in the code's main directory. 

\section{Grid Types}
\label{sec:gridtypes}
\code{GR1D} has several pre-programmed grid arrangements we find work
well for different situations.  \code{GR1D} fully supports non-equally
spaced grids and defines all variables at cell centers.  Available
grids through the {\tt gridtype} parameter are:

\begin{tabular}{lr}
\emph{unigrid}&\parbox[t]{13cm}{Devides the domain into zones of equal
  size.}\\ 
\hline
\emph{log}&\parbox[t]{13cm}{Logarthmic progression of the zone size
  starting from the center out to the domain edge.  The central grid
  spacing in 'log' is set to the parameter {\tt
    grid\_custom\_dx1}}\\ 

\hline

\emph{custom}&\parbox[t]{13cm}{Sets up a region of constant zone width
  near the origin and a logorthmic progression outside.  The value for
  the inner region zone widths is set through {\tt grid\_custom\_dx1}
  and the extent of this inner region is set through {\tt
    grid\_custom\_rad1}.}\\

\hline

\emph{custom2}&\parbox[t]{13cm}{Follows 'custom' closely, but the {\tt
    grid\_custom\_number} innermost zones have zone widths
  logorithically increasing to a value of {\tt grid\_custom\_inner} in
  the innermost zone.  }\\

\end{tabular}

\section{Restart Files}
\code{GR1D} can create restart files that can be used to restart a
simulation at a later time.  These files are in \code{HDF5} format.
The files are checked for consistency of key parameters between
original run and restart. Hence, it is important that the parameter
file of the restarted run is the same as the original. If you happen
to need to change these parameters between restarts, please comment
out the parameter checks.  Older restart files may not work with
updated version of GR1D.

\section{$Y_e(\rho)$}
\label{sec:yeofrhofit}
\code{GR1D} implements a parametrized $Y_e(\rho)$ for the prebounce
evolution or stellar cores.  \code{GR1D} can interpolate a numerical
profile or use the fitting formula of \cite{liebendoerfer:05fakenu}.  The
fitting constants are set in the parameter file.

\section{Rotation}
If rotation is included, an option of analytically setting the angular
velocity profile is available through the following formula,
\begin{equation}
\Omega = \Omega_c / \left[ 1 + \left({r \over A}\right)^2\right]
\label{eq:omega}
\end{equation}
$\Omega_c$ and $A$ are set in the parameter file.  In addition to
this, $\Omega$ can be read in from the initial stellar profile.

\section{Neutrino Transport}

As of version 2, \code{GR1D} now has a neutrino transport module.  The
associated paper is \cite{oconnor:14} and is the main reference for
all of the details. Below we describe the possible parameter file settings.

\section{Parameter File}
Here we list all the options in the parameter file and give a brief
description. 

\begin{itemize}
\item{{\bf Job Parameters}}
\begin{itemize}
\item{{\tt jobname}: text, Description of job.}
\item{{\tt GR}: integer, \code{1}: for GR, \code{0}: for Newtonian}
\item{{\tt outdir}: text, location of stored files, must exist, {\it
    Data/} suggested}
\item{{\tt initial\_data}: text, type of simulation: 'Collapse',
  'OSC', 'Sedov', 'Shocktube','M1test'}
\item{{\tt profile\_name}: text, location of input profile used in 'Collapse' simulations}
\item{{\tt profile\_type}: integer, \code{1}: for .short format}
\item{{\tt gravity\_active}: integer, \code{0}: for no gravity, \code{1}: for gravity}
\item{{\tt ntmax}: integer, maximum hydro steps}
\item{{\tt tend}: real, maximum simulation time (seconds)}
\end{itemize}
\item{{\bf Grid Parameters}}
\begin{itemize}
\item{{\tt geometry}: integer, \code{1}: for linear grid, \code{2}: for spherical}
\item{{\tt gridtype}: text, type of grid arrangement: 'unigrid', 'log', 'custom', 'custom2'.  See \ref{sec:gridtypes} for description}.
\begin{itemize}
\item{{\tt grid\_custom\_dx1}: real, (cm)}
\item{{\tt grid\_custom\_rad1}: real, (cm)}
\item{{\tt grid\_custom\_number}: integer}
\item{{\tt grid\_custom\_inner}: real, (cm)}
\end{itemize}
\item{{\tt rmax\_from\_profile}: integer, \code{0}: specify rmax in {\tt grid\_rmax}, \code{1}: set rmax from density in {\tt rho\_cut}}.
\item{{\tt rho\_cut}: real, density to stop grid at (g/cm$^3$)}.
\item{{\tt grid\_rmax}: real, radius to stop grid at (cm)}
\item{{\tt radial\_zones}: integer, number of radial zones}
\item{{\tt ghosts1}: integer, number of ghost zones}
\end{itemize}
\item{{\bf Hydro Parameters}}
\begin{itemize}
\item{{\tt do\_hydro}: integer, \code{0}: no hydrodynamic step \code{1}: do hydrodynamic step}
\item{{\tt cffac}: real, Courant factor}
\item{{\tt iorder\_hydro}: integer, order of Runge-Kutta}
\item{{\tt reconstruction\_method}: text, reconstruction method: 'ppm', 'tvd', 'pc'}
\item{{\tt ppm\_origin\_TVD}: integer, how many zones of TVD at the origin}
\item{{\tt tvd\_limiter}: text, TVD limiter: 'MC', 'minmod'}
\item{{\tt flux\_type}: text, Riemann solver, only 'HLLE'}
\end{itemize}
\item{{\bf EOS Parameters}}
\begin{itemize}
\item{{\tt eoskey}: integer, type of EOS: \code{1}: Hybrid EOS, \code{2}: Polytrope EOS, \code{3}: Nuclear EOS, \code{4}: Gamma-Law EOS}
\item{{\tt eos\_table\_name}: text, location of {\it .h5} Nuclear EOS table}
\item{{\tt hybridgamma\_th}: real, Hybrid EOS thermal index $\Gamma_{th}$}
\item{{\tt hybridgamma1}: real, Hybrid EOS low density index $\Gamma_{1}$}
\item{{\tt hybridgamma2}: real, Hybrid EOS high density index $\Gamma_{2}$}
\end{itemize}
\item{{\bf Output Parameters}}
\begin{itemize}
\item{{\tt ntinfo}: integer, prints info to stdout after this many hydro steps}
\item{{\tt dynamic\_output\_control}: integer, \code{0}: output frequency stay constant, \code{1}: increases output near bounce and black hole formation}.
\item{{\tt vs\_mass}: integer, \code{0}: for {\it .xg} files versus radial coordinate, \code{1}: for {\it .xg} files versus enclosed baryonic mass}
\item{{\tt small\_output}: integer, \code{0}: output all {\it .xg} files, \code{1}: output smaller subset of {\it .xg} files}
\item{{\tt dtout}: real, time between output of {\it .xg} files (seconds)}
\item{{\tt dtout\_scalar}: real, time between output of {\it .dat} files (seconds)}
\item{{\tt ntout}: integer, hydro steps between output of {\it .xg} files, (-1 to ignore)}
\item{{\tt ntout\_scalar}: integer, hydro steps between output of {\it .dat} files (-1 to ignore)}
\end{itemize}
\item{{\bf Restart Parameters}}
\begin{itemize}
\item{{\tt ntout\_restart}: integer, hydro steps between output of {\it .h5} restart files (-1 to ignore)}
\item{{\tt dtout\_restart}: real, time between output of {\it .h5} restart files (seconds)}
\item{{\tt do\_restart}: integer, \code{0}: no restart, \code{1}: restart}
\item{{\tt restart\_file\_name}: text, location of {\it .h5} file to restart from}
\end{itemize}
\item{{\bf M1 Parameters}}
\begin{itemize}
\item{{\tt do\_M1}: integer, 0 disables transport, 1 enables it}
\item{{\tt v\_order}: integer, -1 full velocity dependence, 0 for none}
\item{{\tt extraction\_radii}: real, radii to evolve neutrinos out to and where data is extracted}
\item{{\tt number\_species}: integer, number of neutrino species, currently only 3 works without minimal adjustment}
\item{{\tt number\_groups}: integer, number of neutrino energy groups, must match table}
\item{{\tt opacity\_table}: string, path to nulib opacity table}
\item{{\tt number\_eas}: integer, number of variables in the eas dataset of the nulib table, usually 3}
\item{{\tt M1closure}: string, denotes closure type, only a couple implemented}
\item{{\tt testcases}: integer, denotes the number of testcase to perform, initial\_data must be 'M1test'}
\item{{\tt include\_epannihil\_kernels}: integer, 1 for including nux pair produciton with the kernels (triggers read from table)}
\item{{\tt include\_nes\_kernels}: integer, 1 for including neutrino electron scattering with kernels (triggers read from table)}
\item{{\tt nes\_evolution\_type}: integer, 0 for no inelastic scattering, 1 for including neutrino electron scattering explicitly, 2 for implicitly}
\item{{\tt energy\_coupling\_type}: integer, 0 for no energy coupling, 1 for including energy coupling explicitly, 2 for implicitly}
\item{{\tt M1\_control}: integer, 0 for no control, parameters stay constant. 1 for primative control, parameters change as below}
\begin{itemize}
\item{{\tt M1\_phase1phase2\_density}: real, central density that triggers phase 2}
\item{{\tt M1\_phase2phase3\_pbtime}: real, post bounce time that triggers phase 3}
\item{{\tt M1\_phase1\_reconstruction}: string, reconstruction method for both matter and neutrinos (phase 1, value must match the value given above)}
\item{{\tt M1\_phase2\_reconstruction}: string, reconstruction method for both matter and neutrinos (phase 2)}
\item{{\tt M1\_phase3\_reconstruction}: string, reconstruction method for both matter and neutrinos (phase 3)}
\item{{\tt M1\_phase1\_cffac}: real, Courant factor, (phase 1)}
\item{{\tt M1\_phase2\_cffac}: real, Courant factor, (phase 2, typically needs to be reduced near bounce)}
\item{{\tt M1\_phase3\_cffac}: real, Courant factor, (phase 3)}
\item{{\tt M1\_phase1\_ns}: integer, number of species to evolve (phase 1; 1 is only nue, 3 is all three)}
\item{{\tt M1\_phase2\_ns}: integer, number of species to evolve (phase 2; 1 is only nue, 3 is all three)}
\item{{\tt M1\_phase3\_ns}: integer, number of species to evolve (phase 3; 1 is only nue, 3 is all three)}
\item{{\tt M1\_phase1\_ies\_way}: integer, 0 for no inelastic scattering, 1 for including neutrino electron scattering explicitly, 2 for implicitly (phase 1, value must match the value given above)}
\item{{\tt M1\_phase2\_ies\_way}: integer, 0 for no inelastic scattering, 1 for including neutrino electron scattering explicitly, 2 for implicitly (phase 2)}
\item{{\tt M1\_phase3\_ies\_way}: integer, 0 for no inelastic scattering, 1 for including neutrino electron scattering explicitly, 2 for implicitly (phase 3)}
\item{{\tt M1\_phase1\_encpl\_way}: integer, 0 for no energy coupling, 1 for including energy coupling explicitly, 2 for implicitly (phase 1, value must match the value given above)}
\item{{\tt M1\_phase2\_encpl\_way}: integer, 0 for no energy coupling, 1 for including energy coupling explicitly, 2 for implicitly (phase 2)}
\item{{\tt M1\_phase3\_encpl\_way}: integer, 0 for no energy coupling, 1 for including energy coupling explicitly, 2 for implicitly (phase 3)}
\end{itemize}
\end{itemize}
\item{{\bf Leakage Parameters}}
\begin{itemize}
\item{{\tt fake\_neutrinos}: integer, \code{0}: for no neutrino physics, \code{1}: for neutrino physics}
\item{{\tt ye\_of\_rho}: integer, \code{0}: no $Y_e(\rho)$, \code{1}: use $Y_e(\rho)$ prescription}
\begin{itemize}
\item{{\tt ye\_profile\_name}: text, profile name of numerical $Y_e(\rho)$ table to interpolate}
\item{{\tt do\_yeofrhofit}: integer, \code{0}: use profile \code{1}: use fit values, see section \ref{sec:yeofrhofit}}
\item{{\tt yeofrho\_rho1}: real, lower density limit of $Y_e(\rho)$ fit, (g/cm$^3$)}
\item{{\tt yeofrho\_rho2}: real, upper density limit of $Y_e(\rho)$ fit, (g/cm$^3$)}
\item{{\tt yeofrho\_ye1}: real, $Y_e$ at lower density limit}
\item{{\tt yeofrho\_ye2}: real, $Y_e$ at upper density limit}
\item{{\tt yeofrho\_yec}: real, $Y_e$ correction factor}
\end{itemize}
\item{{\tt neutrino\_pressure}: integer, \code{0}: no neutrino pressure, \code{1}: include neutrino pressure}
\item{{\tt do\_leak\_ros}: integer, \code{0}: no neutrino leakage/heating, \code{1}: include neutrino leakage/heating}
\item{{\tt do\_heating}: integer, \code{0}: only neutrino leakage, \code{1}: include heating}
\item{{\tt heat\_fac}: real, ad-hoc factor for scaling heating}
\item{{\tt do\_NNBrem}: integer, \code{0}: do not include Nucleon-Nucleon Bremsstrahlung, \code{1}: include it}
\end{itemize}
\item{{\bf Atmosphere Parameters}}
\begin{itemize}
\item{{\tt atmo\_rho\_rel\_min}: real, sets the minimum relative value of the density for the atmosphere (fraction).}
\item{{\tt atmo\_rho\_abs\_min}: real, sets the absolute value of the atmosphere density, recommended value, $\sim$10\% less then {\tt rho\_cut}.}
\item{{\tt atmo\_fac}: real, fraction of atmosphere density to above densities.} 
\end{itemize}
\item{{\bf Rotation}}
\begin{itemize}
\item{{\tt do\_rotation}: integer, \code{0}: no rotation, \code{1}: rotation included}
\item{{\tt set\_omega}: integer, \code{0}: use rotation from {\it .short} file, \code{1}: analytically set omega through Eq.~\ref{eq:omega}}
\item{{\tt omega\_c}: real, central value of omega in Eq.~\ref{eq:omega}}
\item{{\tt omega\_A}: real, radial factor in Eq.~\ref{eq:omega}}
\end{itemize}
\item{{\bf Test problems}}
\begin{itemize}
\item{{\tt shocktube\_problem}: integer, various initial shocktube settings, see {\it src/shocktube.F90}}
\end{itemize}
\end{itemize}

\section{Version 1.01, 1.02, 1.03, \& v2.0 Updates}
\subsection{Physics Updates}
\begin{itemize}
\item We correct the source terms for the neutrino leakage ({\it
    src/leakage\_rosswog/leak\_rosswog.F90}) and neutrino pressure
  contributions ({\it src/neutrino\_pressure.F90}) to the evolution
  equations.
\item v1.01 now includes Newtonian neutrino leakage
\item v1.0 had issues when leakage was preformed near the outer edge
  of the grid (for example with neutron stars).  We have fixed all the
  loop bounds and initialized previously uninitialized variables. 
\item We correct the Newtonian source terms for neutrino pressure and
  leakage (v1.0 missing $\sqrt{\gamma}=r^2$)
\item For {\tt GR}=1, v1.01 now fully includes the effect of the
  moving fluid on the redshifted luminosity of the neutrinos ({\it
    src/leakage\_rosswog/leak\_rosswog.F90}).
\item {\emph{custom2}} grid setup had a bug in v1.0 that would put the
  switch between constant grid and log grid at larger radii then
  requested.
\item v1.01 leakage scheme now uses the neutron rest mass as the
  conversion between $\rho$ and baryon density instead of the atomic
  mass unit.  This is to be more consistant with our EOS tables. 
\item $T/|W_{grav}|$ and $J$ were defined wrong in v1.0, factors of
  $2/3 = \int \sin(\theta)^3 d\theta / \int \sin(\theta)d\theta$ to
  account for averages over the sphere.
\item $T/|W_{grav}|$ now defined for Newtonian simulations.
\item $T/|W_{grav}|$ is now calculated as a function of $r$ and output
  both as an {\emph{.xg}} file and a time series file
  ({\emph{ToverW\_edge.dat}}, with the value at the grid's outer edge).
\item v1.01 will convert the mass variable in the initial data to
  solar masses if not already.
\item v1.01 fixes a reconstruction bug where the temperature was being
  set to the value at the interface, v1.02 fixed a bug that wasn't
  apparent because of this reconstruction bug, keytemp=0 is now used
  exclusively to maintain accuracy in the eps variable.
\item From tests we have been performing, we changed how we evolve our
  angular momentum.  Our conserved variable is now $S_\phi = \rho h W^2
  r v_\varphi$ this, along with the new source terms gives perfect
  angular momentum conservation for Newtonian simulations (as the
  source term is 0) and much better (factor of 40) angular momentum
  conservation for GR runs, as there are still source terms required
  but the largest is now gone.  For Newtonian runs we reconstruct
  $\Omega$ at the interfaces and convert back to $v_\varphi$, for our
  GR runs this leads to problems as we are effectively evolving
  $\Omega r^2$ and near the origin one can evolve to negative
  velocities if the wrong variable is reconstructed.  Hence we
  reconstruct $v_\varphi$ at the interfaces, note the side effect of
  this is a slight divergence of $\Omega$ near the orgin ($r<1$km),
  while visuallly unappealing, this does not contribute to the angular
  momentum as for a constant density sphere the total angular momentum
  is $\propto r^5$. Given this, we recommend using the {\tt
    grid\_type} = 'custom' option for rotating runs which increases
  resolution near the origin.
\item v1.03 now includes the option of include Nucleon-Nucleon
  Bremsstrahlung in the leakage scheme, the appropiate parameter file
  variable must be changed to \code{1} from the default value of
  \code{0}.
\item v2.0 now includes a neutrino transport scheme.
\end{itemize}

\subsection{Useability Updates}
\begin{itemize}
\item In v1.0, some grids would be rejected by the accrection analysis,
in v1.01 this is corrected and \code{GR1D} now gives a warning and
disables the accretion analysis for simulations with $r_{max} <
300$~km this can be editted manually in {\it src/analysis.F90}.
\item Also in the accrection analysis, bus errors could be generated
  for particular grid set ups, this was do to improper initialization
  and has been corrected.
\item In v1.0 the shock radius was determined by locating the $r$
  corresponding to the max absolute velocity.  This method fails for
  exploding stars and now is found locating the $r$ corresponding to
  the minimum velocity which will be the material directly upstream of
  the shock.  As of v1.03, the shock radius and zone index of the
  shock is now included in the restart file.  {\bf{Old restart files
      will NOT be valid as the shock radius values will not be stored.}}
\item In v1.0 the statement 'Using Ye(rho) fit formula' was printed
  for a given set of neutrino parameters but not actually used.
\item The conversion of units for $\Omega$ was done incorrectly for
  the output file {\emph{omega.xg}}.
\item We note that many pre-supernova input files define their
  quantities at different locations (cell interface vs. cell center),
  this can make a difference, for example, in the iron core mass of
  the initial profile if interpolated incorrectly.  \code{GR1D}
  assumes the radial coordinate and density are both cell centered,
  adjustments should be made if this is not the case (e.g., the
  Woosley and Weaver, 1995 and Woosley et al., 2002 presupernova models).
\end{itemize}

{\footnotesize \begin{thebibliography}{104}
\expandafter\ifx\csname natexlab\endcsname\relax\def\natexlab#1{#1}\fi
\expandafter\ifx\csname url\endcsname\relax
  \def\url#1{{\tt #1}}\fi

\bibitem[{O'Connor \& Ott}(2010)]{oconnor:10}
E.~O'Connor \& C.~D.~Ott,
\newblock {\em Class. Quantum Grav.}, 2010 27 114103.

\bibitem[{O'Connor}(2014)]{oconnor:14}
E.~O'Connor,
\newblock {\em submitted to ApJS}, 2014 arXiv:1411.????.

\bibitem[{Liebend{\"o}rfer}(2005)]{liebendoerfer:05fakenu}
M.~{Liebend{\"o}rfer}.
\newblock {\em Astrophys. J.}, {\bf 633}, 1042, 2005.
\end{thebibliography}
\end{document}
